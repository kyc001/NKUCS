\documentclass[12pt,a4paper,UTF8]{ctexart}




%设置页边距
\usepackage{geometry}
\geometry{left=2.5cm,right=2.5cm,top=2.5cm,bottom=2.5cm}




%需要用到的扩展包
\usepackage{xeCJK,amsmath,paralist,enumerate,booktabs,multirow,graphicx,float,subfig,setspace,listings,lastpage,hyperref}
\usepackage{fancyhdr}




%设置页眉页脚以及页码
\pagestyle{fancy}
\rhead{直流双臂电桥}
\lhead{大学基础物理实验报告}
\rfoot{2025 年 4 月 15 日}




%报告中用到的图片存放在这个tex文件所在目录中的figures子目录中
\graphicspath{{figures/}}









%报告开始
\begin{document}
	
	
	
	
	%设置课程标题
	\begin{center}
		\heiti\LARGE{《大学基础物理实验》课程实验报告}
	\end{center}
	
	
	
	
	%设置实验人信息以及实验时间表格


		\begin{center}
			\begin{tabular}{lcr}
				
				{\songti 姓名:柯云超  \quad 学号: 2413575}\\
				{\songti  学院:计算机学院 \quad 时间:2025年4月15日 \quad 组别: L组11号}\\
				
				
			\end{tabular}
		\end{center}
	\vspace{-0.2cm}
	{\noindent}	 \rule[-10pt]{16cm}{0.05em}\\

	\vspace{-0.4cm}
	
	
	
	
	
	
	%实验题目
	\begin{center}
		\LARGE\textbf{直流双臂电桥}
	\end{center}
	

	
	%实验原理
	\subsection*{一、实验原理}
	\subsubsection*{直流双臂电桥适用范围:}
	\par 直流双臂电桥适用于测量较小的电阻,如QJ44型直流双臂电桥测量范围: 0.1m$\Omega$-11$\Omega$
	\subsubsection*{四端法:}
	\begin{figure}[!htbp]
		\centering
		\includegraphics[width=0.6\textwidth]{四端法}
		\caption{四端法原理}
	\end{figure}
	\par 可以看出,使用图1的电路进行测量,在电阻体上$Y,Y^{\prime}$上两个点焊出两个接头再与微安表相连接,这样可以保证微安表所连接两点之间的阻值正好为$Y,Y^{\prime}$之间的阻值,又$A,B,P,P^{\prime}$四个点的接触电阻和$AY,BY^{\prime}$的接线电阻都分给了微安表,保证了分流的精确。由于电阻被做成了四个接头,故称作“四端结构”。
	
	\subsubsection*{实验电路与测量公式推导}
	\par 测量电路如图2(见下页)所示,其中$R_0$为标准低阻,$R_x$为待测低阻。四个比例臂电阻有意做成几十欧姆以上的阻值,因此他们所在桥臂中的接线电阻和接触电阻的影响便可忽略。注意右边的电阻$R$是为了防止电流过大。当电流计指零时,电桥达到平衡。
	\clearpage
	\begin{figure}
		\centering
		\includegraphics[width=0.6\textwidth]{直流双臂电路图透明}
		\caption{直流双臂电路}
	\end{figure}
	\par 由基尔霍夫定律,可以列出方程组:

	\begin{equation}
		\begin{cases}
			$$I_1 R_1=I_0 R_0+I_1^{\prime} R_1^{\prime} \\
			I_1 R_2=I_0 R_x+I_1^{\prime} R_2^{\prime} \\
			\left(I_0-I_1^{\prime}\right) R_{\mathrm{r}}=I_1^{\prime}\left(R_1^{\prime}+R_2^{\prime}\right)$$
		\end{cases}
	\end{equation}
	\par 式中$I_1,I_0,I_1^{\prime}$分别为图中所标示,将(1)式整理得:
	\[R_{1}R_{x}=R_{2}R_{0}+(\,R_{2}R_{1}^{\prime}-R_{1}R_{2}^{\prime}\,)\,{\frac{r}{R_{r}+R_{1}^{\prime}+R_{2}^{\prime}}}\]
	\par 当电桥的平衡是在保证$R_{2}R_{1}^{\prime}-R_{1}R_{2}^{\prime}=0$的情况下,则上式可以简化为
	\[R_{x}={\frac{R_{2}}{R_{1}}}R_{0}\]
	\par 由此可知此次实验双臂电桥的测量平衡条件为:
	\[{\frac{R_{2}}{{R_{1}}}}={\frac{{R_{2}^{\prime}}}{{R_{1}^{\prime}}}}={\frac{{R_{x}}}{{R_{0}}}}\]
	\par 本次实验使用同步调节比例臂电阻$R_2,R_2^{\prime}$的方法使电流计示零。
	\subsubsection*{双臂电桥灵敏度}
	\par 双臂电桥平衡后将比例臂电阻$R_2,R_2^{\prime}$同步调偏$\Delta R_2=\Delta R_2^{\prime}$,若电流计示数改变$\Delta I$,则灵敏度$S$为:
	\[
	S=\frac{\Delta I}{\Delta R_2 / R_2}
	\]
	\clearpage
	\par 由$S={\dfrac{\Delta I}{\Delta R_{2}/R_{2}}}={\dfrac{\Delta I}{\Delta R_{x}/R_{x}}}$可以引入相对误差:
	$${\frac{\Delta R_{x}}{R_{x}}}={\frac{\Delta I}{S}}$$




	\subsection*{二、主要仪器用具}
        \par 仪器品牌和型号
        \par 电源: DF1709SB 
        \par 电流检测仪器: FB3082型直流数显微电流计 \quad 分辨率:0.1nA \quad 量程:200nA
        \par 标准低阻 $R_0$: 阻值 0.001 $\Omega$
        \par 固定电阻 $R_1 = R_1'$: 阻值 1000 $\Omega$
        \par 电阻箱型号: FBZX21型直流电阻箱






    
	\subsection*{三、数据处理}
        
	\subsubsection*{铜棍电阻率的测量}

(1) 铜棍长度(两个电压接头之间):
   注:直尺单次测量\(B\)类不确定度:\(u_{Bx}=\frac{\Delta}{\sqrt{3}}\)(\(\Delta = 0.5mm\))
   数据处理:
   \[ l = (\underline{\text{473.5}} \pm \underline{0.29})mm \]

(2) 铜棍直径测量:
   螺旋测微器零点读数:\(\underline{0}mm\)

   \begin{table}[h]
       \centering
       \caption{铜棍直径测量数据}
       \begin{tabular}{ccccccc}
           \toprule
           测量次数 & 1 & 2 & 3 & 4 & 5 & 平均值 \\
           \midrule
           直径(mm) & 4.965 & 4.965 & 4.965 & 4.970 & 4.970 & \(\boldsymbol{4.967}\) \\
           \bottomrule
       \end{tabular}
   \end{table}

   注:\(A\)类不确定度 \(u_{Ax}=t_{(0.683,k)}S_{\overline{x}}\),\(S_{\overline{x}} = \frac{S_{x_i}}{\sqrt{n}}=\left[\frac{\sum_{i = 1}^{n}(x_{i}-\overline{x})^{2}}{n(n - 1)}\right]^{\frac{1}{2}}\)

   \(B\)类不确定度螺旋测微器分辨率\(\varepsilon_{x}=0.001mm\),多次测量的\(B\)类标准不确定度\(u_{Bx}=\frac{\varepsilon_{x}}{\sqrt{3}}\)。


    \(u_x=\sqrt{u_{ax}^{2}+u_{bx}^{2}}\)

    数据处理:
     \[ d =(\underline{\quad4.967\quad}\pm\underline{\quad0.0015\quad})mm \]

(3) 调节电桥平衡:
    
    \begin{table}[h]
       \centering
       \caption{铜棍电桥测量数据}
       \begin{tabular}{ccccccc}
           \toprule
           电桥状态 & $R_2(=R_2')$ & $R_x$ & $\Delta R_2(=\Delta R_2')$ & $\Delta I$ & S \\
           \midrule
           数据记录 & 368 & $3.68\times10^{-4}$ & 20 & 3.6 & 66.24 \\
           \bottomrule
       \end{tabular}
   \end{table}


\(R_x\) 的总相对不确定度为 \(\rho_{x}=\sqrt{(1 + k)^{2}(\rho_{2}^{2}+\rho_{1}^{2})+k^{2}(\rho_{2}^{'2}+\rho_{1}^{'2})+\rho_{0}^{2}+(0.1}/S)^{2}\)

其中 \(\rho_{1}=\rho_{1}'=\rho_{2}=\rho_{2}' = 0.1\%\),\(\rho_{0}=0.05\%\),\(k = 0.1\),\(u_{Rx}=\rho_{x}R_{x}\)

数据处理:
将已知数值代入公式:


\begin{align*}
\rho_{x}&=\sqrt{(1+0.1)^{2}*(0.001^{2}+0.001^{2})+0.1^{2}*(0.0001^{2}+0.001^{2})+0.0005^{2}+(0.1/66.24)^{2}}\\
&\approx0.002\\
&= 0.2\%
\end{align*}


则电阻值 \(R_{x}=(\underline{\quad 3.68\quad}\pm\underline{\quad 0.007\quad})\times10^{-4}\Omega\)

(4) 电阻率

注:\(\rho_{Rx}=R_{x}S/L=\pi R_{x}d^{2}/4L\),求全微分得
\[ u_{\rho}=\rho\left[(u_{R}/R)^{2}+(2u_{d}/d)^{2}+(u_{l}/L)^{2}\right]^{1/2} \]

数据处理:

\[\rho_{Rx}=\pi\times3.68\times 10 ^{-4} \times (4.967\times10^{-3})^{2}/(4\times473.5\times10^{-3})= 1.505\times10^{-8}\]

\[ u_{\rho}=1.505\times 10^{-8} \times \left[(0.002)^{2}+(0.003/4.967)^{2}+(0.29/473.5)^{2}\right]^{1/2} = 0.003\times10^{-8} \]


电阻率 \(= (\underline{\quad1.505\quad}\pm\underline{\quad0.003\quad})\times10^{-8}\Omega\cdot m\)


	\subsubsection*{铝棍电阻率的测量}

(1) 铝棍长度(两个电压接头之间):
   注:直尺单次测量\(B\)类不确定度:\(u_{Bx}=\frac{\Delta}{\sqrt{3}}\)(\(\Delta = 0.5mm\))
   数据处理:
   \[ l = (\underline{\text{475.5}} \pm \underline{0.29})mm \]

(2) 铝棍直径测量:
   螺旋测微器零点读数:\(\underline{0}mm\)

   \begin{table}[h]
       \centering
       \caption{铝棍直径测量数据}
       \begin{tabular}{ccccccc}
           \toprule
           测量次数 & 1 & 2 & 3 & 4 & 5 & 平均值 \\
           \midrule
           直径(mm) & 4.935 & 4.935 & 4.940 & 4.935 & 4.935 & \(\boldsymbol{4.936}\) \\
           \bottomrule
       \end{tabular}
   \end{table}

   注:\(A\)类不确定度 \(u_{Ax}=t_{(0.683,k)}S_{\overline{x}}\),\(S_{\overline{x}} = \frac{S_{x_i}}{\sqrt{n}}=\left[\frac{\sum_{i = 1}^{n}(x_{i}-\overline{x})^{2}}{n(n - 1)}\right]^{\frac{1}{2}}\)

   \(B\)类不确定度螺旋测微器分辨率\(\varepsilon_{x}=0.001mm\),多次测量的\(B\)类标准不确定度\(u_{Bx}=\frac{\varepsilon_{x}}{\sqrt{3}}\)。


    \(u_x=\sqrt{u_{ax}^{2}+u_{bx}^{2}}\)

    数据处理:
     \[ d =(\underline{\quad4.936\quad}\pm\underline{\quad0.0013\quad})mm \]

(3) 调节电桥平衡:
    
    \begin{table}[h]
       \centering
       \caption{铝棍电桥测量数据}
       \begin{tabular}{ccccccc}
           \toprule
           电桥状态 & $R_2(=R_2')$ & $R_x$ & $\Delta R_2(=\Delta R_2')$ & $\Delta I$ & S \\
           \midrule
           数据记录 & 895 & $8.95\times10^{-4}$ & 25 & 3.0 & 161.1 \\
           \bottomrule
       \end{tabular}
   \end{table}


\(R_x\) 的总相对不确定度为 \(\rho_{x}=\sqrt{(1 + k)^{2}(\rho_{2}^{2}+\rho_{1}^{2})+k^{2}(\rho_{2}^{'2}+\rho_{1}^{'2})+\rho_{0}^{2}+(0.1}/S)^{2}\)

其中 \(\rho_{1}=\rho_{1}'=\rho_{2}=\rho_{2}' = 0.1\%\),\(\rho_{0}=0.05\%\),\(k = 0.1\),\(u_{Rx}=\rho_{x}R_{x}\)

数据处理:
将已知数值代入公式:


\begin{align*}
\rho_{x}&=\sqrt{(1+0.1)^{2}*(0.001^{2}+0.001^{2})+0.1^{2}*(0.001^{2}+0.001^{2})+0.0005^{2}+(0.1/161.1)^{2}}\\
&\approx0.0018\\
&= 0.18\%
\end{align*}


则电阻值 \(R_{x}=(\underline{\quad 8.95\quad}\pm\underline{\quad 0.016\quad})\times10^{-4}\Omega\)

(4) 电阻率

注:\(\rho_{Rx}=R_{x}S/L=\pi R_{x}d^{2}/4L\),求全微分得
\[ u_{\rho}=\rho\left[(u_{R}/R)^{2}+(2u_{d}/d)^{2}+(u_{l}/L)^{2}\right]^{1/2} \]

数据处理:

\[\rho_{Rx}=\pi\times8.95\times 10 ^{-4} \times (4.936\times10^{-3})^{2}/(4\times475.5\times10^{-3})= 3.662\times10^{-8}\]

\[ u_{\rho}=3.662\times 10^{-8} \times \left[(0.0018^{2}+(0.0026/4.936)^{2}+(0.29/475.5)^{2}\right]^{1/2} = 0.006\times10^{-8} \]


电阻率 \(= (\underline{\quad3.662\quad}\pm\underline{\quad0.006\quad})\times10^{-8}\Omega\cdot m\)

	\subsubsection*{铁棍电阻率的测量}

(1) 铁棍长度(两个电压接头之间):
   注:直尺单次测量\(B\)类不确定度:\(u_{Bx}=\frac{\Delta}{\sqrt{3}}\)(\(\Delta = 0.5mm\))
   数据处理:
   \[ l = (\underline{\text{475.0}} \pm \underline{0.29})mm \]

(2) 铁棍直径测量:
   螺旋测微器零点读数:\(\underline{0}mm\)

   \begin{table}[h]
       \centering
       \caption{铁棍直径测量数据}
       \begin{tabular}{ccccccc}
           \toprule
           测量次数 & 1 & 2 & 3 & 4 & 5 & 平均值 \\
           \midrule
           直径(mm) & 4.975 & 4.975 & 4.985 & 4.975 & 4.975 & \(\boldsymbol{4.977}\) \\
           \bottomrule
       \end{tabular}
   \end{table}

   注:\(A\)类不确定度 \(u_{Ax}=t_{(0.683,k)}S_{\overline{x}}\),\(S_{\overline{x}} = \frac{S_{x_i}}{\sqrt{n}}=\left[\frac{\sum_{i = 1}^{n}(x_{i}-\overline{x})^{2}}{n(n - 1)}\right]^{\frac{1}{2}}\)

   \(B\)类不确定度螺旋测微器分辨率\(\varepsilon_{x}=0.001mm\),多次测量的\(B\)类标准不确定度\(u_{Bx}=\frac{\varepsilon_{x}}{\sqrt{3}}\)。


    \(u_x=\sqrt{u_{ax}^{2}+u_{bx}^{2}}\)

    数据处理:
     \[ d =(\underline{\quad4.977\quad}\pm\underline{\quad0.0017\quad})mm \]

(3) 调节电桥平衡:
    
    \begin{table}[h]
       \centering
       \caption{铁棍电桥测量数据}
       \begin{tabular}{ccccccc}
           \toprule
           电桥状态 & $R_2(=R_2')$ & $R_x$ & $\Delta R_2(=\Delta R_2')$ & $\Delta I$ & S \\
           \midrule
           数据记录 & 15110 & $151.1\times10^{-4}$ & 200 & 2.6 & 196.43 \\
           \bottomrule
       \end{tabular}
   \end{table}


\(R_x\) 的总相对不确定度为 \(\rho_{x}=\sqrt{(1 + k)^{2}(\rho_{2}^{2}+\rho_{1}^{2})+k^{2}(\rho_{2}^{'2}+\rho_{1}^{'2})+\rho_{0}^{2}+(0.1}/S)^{2}\)

其中 \(\rho_{1}=\rho_{1}'=\rho_{2}=\rho_{2}' = 0.1\%\),\(\rho_{0}=0.05\%\),\(k = 0.1\),\(u_{Rx}=\rho_{x}R_{x}\)

数据处理:
将已知数值代入公式:


\begin{align*}
\rho_{x}&=\sqrt{(1+0.1)^{2}*(0.001^{2}+0.001^{2})+0.1^{2}*(0.001^{2}+0.001^{2})+0.0005^{2}+(0.1/196.43)^{2}}\\
&\approx0.0017
\\
&= 0.17\%
\end{align*}


则电阻值 \(R_{x}=(\underline{\quad 151.10\quad}\pm\underline{\quad 0.26\quad})\times10^{-4}\Omega\)

(4) 电阻率

注:\(\rho_{Rx}=R_{x}S/L=\pi R_{x}d^{2}/4L\),求全微分得
\[ u_{\rho}=\rho\left[(u_{R}/R)^{2}+(2u_{d}/d)^{2}+(u_{l}/L)^{2}\right]^{1/2} \]

数据处理:

\[\rho_{Rx}=\pi\times151.0\times 10 ^{-4} \times (4.977\times10^{-3})^{2}/(4\times475.0\times10^{-3})= 6.189\times10^{-7}\]

\[ u_{\rho}=6.189\times 10^{-7} \times \left[(0.0017)^{2}+(0.0034/4.977)^{2}+(0.29/475.0)^{2}\right]^{1/2} = 0.011\times10^{-7} \]

电阻率 \(= (\underline{\quad6.189\quad}\pm\underline{\quad0.011\quad})\times10^{-7}\Omega\cdot m\)

\subsection*{四、注意事项}

1.连好电路复查无误,$R_2$以及$R_2'$ (\approx $R_2$)的阻值设置好,预先使桥接近平衡,才允许试接电源进行实验。其中对于铜棍,$R_2$,$R_2'$预置在 500Ω ,对于铝棍,$R_2$,$R_2'$预置在 1000Ω ,对于铁棍,$R_2$,$R_2'$预置在 15000Ω 。接入电路中的棍的长度要大于 40cm,以减少误差。
 
2.电路连接时注意长线短线的选择。
 
3.开启电源前,电流旋钮调到最大位置,电压旋钮调到最小位置。
 
4.接通电源后,观察直流稳压电源输出电压变化,如果电压降低(电路出现短路情况),立刻切断电源,查找原因。
 
5.注意调节电桥平衡时,应逐渐增大电源电压,先在 1V 时调节平衡,然后逐渐增加到 2V,3V。
 
6.调节平衡时,在看电流偏转大小时才短暂接通一下电路,不可在接通电路情况下,调节$R_2$,$R_2'$。接通电源时间应尽量短,以免电阻发热。
 
7.直流数字微电流表使用之前调零校准。
 
8.实验结束后,电阻箱阻值不要归零。


\subsection*{五、实验分析讨论及思考题。(思考题2)}

若均匀板状低阻上电流线的分布如图所示,那么在测低阻材料的电阻率时,应该测哪两条线之间的电阻?如选择不当,测出的电阻率偏大还是偏小?

	\begin{figure}[h]
		\centering
		\includegraphics[width=0.6\textwidth]{figures/图片1.jpg}
		\caption{思考题2}
	\end{figure}
 
应测量B、C之间的电阻。若选择不当,由于在A和B、C和D之间的电流经过的电阻横截面比电阻的整个横截面面积更小,因此这一段由原本的横截面积和所测得的电阻而计算得出的电阻率会比原本的电阻率偏大。



\end{document}
