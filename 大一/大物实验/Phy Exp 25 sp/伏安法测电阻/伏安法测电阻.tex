\documentclass[12pt,a4paper,UTF8]{ctexart}




%设置页边距
\usepackage{geometry}
\geometry{left=2.5cm,right=2.5cm,top=2.5cm,bottom=2.5cm}


\usepackage{subcaption} % 用于添加子标题和并排排版表格
\usepackage{booktabs} % 提供更美观的表格线,这里可根据实际情况选择是否保留

%需要用到的扩展包
\usepackage{xeCJK,amsmath,paralist,enumerate,booktabs,multirow,graphicx,float,subfig,setspace,listings,lastpage,hyperref}
\usepackage{fancyhdr}




%设置页眉页脚以及页码
\pagestyle{fancy}
\rhead{伏安法测定电阻}
\lhead{大学基础物理实验报告}

\rfoot{2025 年 3 月 25 日}




%报告中用到的图片存放在这个tex文件所在目录中的figures子目录中
\graphicspath{{figures/}}









%报告开始
\begin{document}
	
	
	
	
	%设置课程标题
	\begin{center}
		\heiti\LARGE{《大学基础物理实验》课程实验报告}
	\end{center}
	
	
	
	
	%设置实验人信息以及实验时间表格


		\begin{center}
			\begin{tabular}{lcr}
				
				{\songti 姓名:柯云超  \quad 学号: 2413575}\\
				{\songti  学院:计算机学院 \quad 时间:2025年3月25日 \quad 组别: L组11号}\\
				
				
			\end{tabular}
		\end{center}
	\vspace{-0.2cm}
	{\noindent}	 \rule[-10pt]{16cm}{0.05em}\\

	\vspace{-0.4cm}
	
	
	
	
	
	
	%实验题目
	\begin{center}
		\LARGE\textbf{伏安法测定电阻}
	\end{center}
	

	
	%实验原理
	\subsection*{[实验原理]}
	\subsubsection*{线性元件和非线性元件}
	\par 伏安特性曲线是以电流$U$为横坐标,电压$I$为纵坐标作出的I-U图。通常情况下,导电金属丝,碳膜电阻,金属膜电阻等的伏安特性曲线是一条直线,称其为线性元件;二极管等元件的伏安特性曲线不是一条直线,这类元件称为非线性元件。
	\vspace{0.2cm}
	\begin{figure}[htbp]
		\centering
		\begin{minipage}[t]{0.48\textwidth}
			\centering
			\includegraphics[width=5cm]{线性元件}
			\caption{线性元件}
		\end{minipage}
		\begin{minipage}[t]{0.48\textwidth}
			\centering
			\includegraphics[width=5cm]{非线性元件}
			\caption{非线性元件}
		\end{minipage}
	\end{figure}
	\vspace{-0.8cm}
	\subsubsection*{测量电路}
	\par 本实验中测量电路分为两种:分压式和限流式,本次实验选择分压电路,电压表内接
		\begin{figure}[H]
		\centering
		\begin{minipage}[t]{0.48\textwidth}
			\centering
			\includegraphics[width=6cm]{分压电路}
			\caption*{(a)分压电路}
		\end{minipage}
		\begin{minipage}[t]{0.48\textwidth}
			\centering
			\includegraphics[width=6cm]{限流电路}
			\caption*{(b)限流电路}
		\end{minipage}
	\caption{测量电路选取}
	\end{figure}
	\vspace{0.2cm}
	


	\subsection*{[主要仪器品牌与型号]}
	\par 直流稳压电源
	\par 台式万用表:GDM8342
	\par 手持万用表:VT61B
	\par 滑动变阻器:BX7-11
	\vspace{0.2cm}




        \subsection*{[万用表测量数据]}
	\par 金属膜电阻$R_x$阻值:109.1$\Omega$
	\par 直流稳压电源输出电压:1.49$V$
	\par 正向$PN$结电压:0.385mV
        \par 电表内阻$R_V = 10M\Omega$,  $R_A = 2\Omega$, 电压表应内接

	\vspace{0.2cm}

	
	
	
	
	
	
	
	
	%实验内容与分析
	\subsection*{[伏安法测量数据]}
	\par 实验测量数据如下
\begin{table}[!h]
    \centering
    \begin{tabular}{|l|l|l|l|l|l|l|}
        \hline
        $U(V)$ & 0.0055 & 0.0723 & 0.2110 & 0.2277 & 0.3918 & 0.4382 \\
        \hline
        $I(mA)$ & 0.04 & 0.66 & 1.93 & 2.09 & 3.56 & 3.92 \\
        \hline
        $U(V)$ & 0.4686 & 0.5640 & 0.7081 & 0.9434 & 1.1782 & 1.4032 \\
        \hline
        $I(mA)$ & 4.29 & 5.16 & 6.49 & 8.63 & 10.87 & 12.85 \\
        \hline
    \end{tabular}
        \caption{金属膜电阻原始数据}
\end{table}

\begin{table}[!h]
    \centering
    \begin{tabular}{|l|l|l|l|l|l|l|}
        \hline
        $U(V)$ & 0.0150 & 0.1451 & 0.2308 & 0.3104 & 0.4653 & 0.5440 \\
        \hline
        $I(mA)$ & 0.00 & 0.00 & 0.00 & 0.01 & 0.23 & 1.73 \\
        \hline
        $U(V)$ & 0.5851 & 0.6046 & 0.6399 & 0.6552 & 0.6775 & 0.7056 \\
        \hline
        $I(mA)$ & 4.00 & 5.72 & 10.44 & 13.47 & 19.42 & 31.92 \\
        \hline
    \end{tabular}
        \caption{二极管原始数据}
\end{table}

	\vspace*{-0.6cm}

         \newpage
   
	\subsection*{[数据处理]}
        \par 使用Python绘制伏安特性曲线如下.
            \begin{figure}[H]
		\centering
		\includegraphics[width=0.7\linewidth]{figures/金属膜电阻.png}
		\caption{线性电阻直线拟合}
            \end{figure}
            \begin{figure}[H]
		\centering
		\includegraphics[width=0.7\linewidth]{figures/二极管.png}
		\caption{非线性电阻直线拟合}
            \end{figure}

	\subsubsection*{金属膜电阻数据处理}
	\par 选取直线上距离最远的两个点,带入公式
	\[\overline{R_x}=\frac{U_2-U_1}{I_2-I_1-\dfrac{U_2-U_1}{R_V}}
	\]
	\par 计算得$\overline{R}_x=\dfrac{1.40-0.01}{12.85-0.04-\dfrac{1.40-0.01}{10^7}}\times1000\approx108.51\Omega$
        
        \par 下面进行误差分析:
	\par 首先计算仪表误差:
	\[\Delta U = \pm (0.0002*1.40+4*0.0001) \]
	\[\Delta I = \pm (0.012*12.85+3*0.01) \]
	\par 计算相对误差$\rho_x$
	\[\rho_x=\sqrt{\rho_V^2+\rho_z^2}=\sqrt{\left(\frac{\Delta U}{U_2-U_1}\right)^2+\left(\frac{\Delta I}{I_2-I_1}\right)^2}=0.014\]
	\par 则误差为$\Delta R = \overline{R}_x\times \rho_x = 1.52\Omega$,求得最终结果为
	\[R_x = (108.51\pm 1.52)\Omega \]
    
	\subsubsection*{二极管数据处理}
	\par 从图中读取数据,根据有效位数"多取一位"原则,保留三位有效数字。
	\par ($a$) 在2.00mA下的阻值为$\frac{0.5507}{2}*1000 = 275 \Omega$
	\par ($b$) 在8.00mA下的阻值为$\frac{0.6241}{8}*1000 = 78.0\Omega$

    
	\subsection*{[思考题]}
	

            分析:三个元件的阻值数量级差别都很大,在现有条件下无论选择怎样的方法都会造成较大的误差。


            \begin{figure}[H]
		\centering
		\includegraphics[width=0.7\linewidth]{figures/思考题2.png}
		\caption{思考题2}
            \end{figure}

            
            按照图中的方法,可以先断开开关,根据$(100+R_G)I_{G1}=U$先求得电路总电压,然后合上开关,利用$\frac{U-I_{G2}R_G}{100}-I_{G2}=\frac{I_{G2}R_G}{R_x}$求得$R_x$。
	
	
	
	
	


\end{document}
