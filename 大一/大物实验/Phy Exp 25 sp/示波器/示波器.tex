\documentclass[12pt,a4paper,UTF8]{ctexart}




%设置页边距
\usepackage{geometry}
\geometry{left=2.5cm,right=2.5cm,top=2.5cm,bottom=2.5cm}




%需要用到的扩展包
\usepackage{xeCJK,amsmath,paralist,enumerate,booktabs,multirow,graphicx,float,subfig,setspace,listings,lastpage,hyperref}
\usepackage{fancyhdr}




%设置页眉页脚以及页码
\pagestyle{fancy}
\rhead{示波器的使用}
\lhead{大学基础物理实验报告}
\rfoot{2025 年 4 月 1 日}




%报告中用到的图片存放在这个tex文件所在目录中的figures子目录中
\graphicspath{{figures/}}









%报告开始
\begin{document}
	
	
	
	
	%设置课程标题
	\begin{center}
		\heiti\LARGE{《大学基础物理实验》课程实验报告}
	\end{center}
	
	
	
	
	%设置实验人信息以及实验时间表格
	
	
		\begin{center}
			\begin{tabular}{lcr}
				
				{\songti 姓名:柯云超  \quad 学号: 2413575}\\
				{\songti  学院:计算机学院 \quad 时间:2025年4月1日 \quad 组别: L组11号}\\
				
				
			\end{tabular}
		\end{center}
	\vspace{-0.2cm}
	{\noindent}	 \rule[-10pt]{16cm}{0.05em}\\

	\vspace{-0.4cm}
	
	
	
	
	
	
	%实验题目
	\begin{center}
		\LARGE\textbf{示波器的使用}
	\end{center}
	
	
	
	%实验原理
	\subsection*{[仪器与用具]}
	\par 1.1仪器品牌与型号:
	\par 示波器:普源DS1102E,信号发生器:F05函数发生器
	\par 1.2电阻阻值:1k$\Omega$,电容值:0.1$\mu F$
	
	\subsection*{[基本使用]}
	\par 将信号源($1KHz,3Vp-p$)和变压器电压同时输出到示波器,分别稳定并显示适当的波形。重点熟悉触发对波形的作用。

	\subsection*{[实验原理]}
        \par 示波器: CRT,亦称阴极射线管,把待测电信号变成发光的图形
        \par 波形显示的基本原理: Y偏转板加$u_y$, 在X偏转板添加锯齿波电压,随时间线性添加到最大值,然后突然回到最小,此后再重复地变化。
        \par 波形稳定的条件: $T_x = nT_y$.

	\subsection*{[实验内容及步骤]}
        \par 1.将信号发生器信号(频率约 1$kHz$,电压峰-峰值约3V)和市电小电压信号(频率约 50 Hz,电压峰 - 峰值约 6 V)同时接到示波器 CH1 和 CH2 接口,分别在示波器屏幕上调节出稳定的波形,熟悉触发对波形的作用。

        \par 2.根据示波器类型,采用自动测量、手动光标测量,或直接读格数等三种方法,测量上述两个信号的电压峰 - 峰值和频率。
        \par 3.以信号源频率为已知,利用李萨如图形测量市电频率。
        \par 4.连接 RC 电路,用双踪显示法和李萨如图法测量该电路输入弦信号 $u_1$ 和输出信号 $u_2$ 之间的相位差,CH1 接 $u_1$ 信号,CH2 接 $u_2$ 信号。信号频率取 f = 1.59 kHz,电容 C = 0.1 μF,电阻 R = 1 kΩ。
    
	\subsection*{[实验数据]}
	\subsubsection*{信号源和变压器测量}
	\begin{table}[!htbp]
		\centering
	\begin{tabular}{|p{2cm}|p{2cm}|p{2cm}|p{2cm}|}
		\hline
		\rule{0pt}{16pt} 信号源 & 自动测量 & 光标测量 & 读格测量 \\ \hline
		\rule{0pt}{16pt} 电压  & 3.10V & 3.12V&3.00V\\ \hline
		\rule{0pt}{16pt} 周期  & 1.004ms  & 1.01ms & 1ms\\ \hline
		\rule{0pt}{16pt} 频率  & 996.01Hz  & 990.09Hz &1000Hz \\ \hline

	\end{tabular}
    		\caption{实验测量数据}

\end{table}

\par 实验图片如下:
            \begin{figure}[H]
		\centering
		\includegraphics[width=0.7\linewidth]{figures/自动测量.jpg}
		\caption{自动测量}
            \end{figure}

            \begin{figure}[H]
		\centering
		\includegraphics[width=0.7\linewidth]{figures/光标测量电压.jpg}
		\caption{光标测量电压}
            \end{figure}

            \begin{figure}[H]
		\centering
		\includegraphics[width=0.7\linewidth]{figures/光标测量周期.jpg}
		\caption{光标测量周期}
            \end{figure}
            
	\subsubsection*{李萨如图形测量法}
        \par 在测量图形前,应该分别将CH1和CH2信号调到稳定,图像大小差不多后再将时基切换至 X-Y 模式。
	\begin{table}[!htbp]
		\centering 
		\begin{tabular}{|c|c|c|c|c|}
			\hline
			\rule{0pt}{20pt}$\frac{\mbox{与水平线的交点数}}{\mbox{与竖直线交点数}}=\frac{n_x}{n_y}$    & 1:1 & 1:2 & 1:3 & 2:3  \\ \hline
			\rule{0pt}{20pt}函数发生器频率 & 50.02 & 100.06 & 150.10 & 75.05   \\ \hline
			\rule{0pt}{20pt}算出的市电频率 &  50.02& 50.03 & 50.03 &50.03 \\ \hline
		\end{tabular}
            \caption{李萨如图形数据}
	\end{table}
	\par 计算得到平均市电频率为50.03Hz.
	\subsection*{[测量RC电路的相位差]}
	\par 连接电路,将信号发生器频率设定为$f=1.59kHz$
	\subsubsection*{椭圆法}
	$$
	|\theta|=\arcsin\frac{2x_0}{2x_m}=\arcsin\frac{2.00}{3.02}\approx0.72
	$$
	\subsubsection*{位移法}
	\[
	\theta = \frac{l}{l_0}\times 2\pi = \frac{76.0}{628}\times 2\pi\approx0.76
	\]

	\subsubsection*{[思考题]}
	\par 无
	
	
	
	
	
	
\end{document}
