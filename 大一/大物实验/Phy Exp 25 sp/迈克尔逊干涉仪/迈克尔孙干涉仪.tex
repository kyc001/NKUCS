\documentclass[12pt,a4paper,UTF8]{ctexart}




%设置页边距
\usepackage{geometry}
\geometry{left=2.5cm,right=2.5cm,top=2.5cm,bottom=2.5cm}
\usepackage{wrapfig}



%需要用到的扩展包
\usepackage{xeCJK,amsmath,paralist,enumerate,booktabs,multirow,graphicx,float,subfig,setspace,listings,lastpage,hyperref}
\usepackage{fancyhdr}
\usepackage{ctex}


%设置页眉页脚以及页码
\pagestyle{fancy}
\rhead{迈克尔孙干涉仪}
\lhead{大学基础物理实验报告}
\rfoot{2025 年 5 月 6 日 }




%报告中用到的图片存放在这个tex文件所在目录中的figures子目录中
\graphicspath{{figures/}}









%报告开始
\begin{document}
	
	
	
	
	%设置课程标题
	\begin{center}
		\heiti\LARGE{《大学基础物理实验》课程实验报告}
	\end{center}
	
	
	
	
	%设置实验人信息以及实验时间表格
	
	
		\begin{center}
			\begin{tabular}{lcr}
				
				{\songti 姓名:柯云超  \quad 学号: 2413575}\\
				{\songti  学院:计算机学院 \quad 时间:2025年5月6日 \quad 组别: L组11号}\\
				
				
			\end{tabular}
		\end{center}
	\vspace{-0.2cm}
	{\noindent}	 \rule[-10pt]{16cm}{0.05em}\\

	\vspace{-0.4cm}
	
	
	
	
	
	
	%实验题目
	\begin{center}
		\LARGE\textbf{迈克尔孙干涉仪}
	\end{center}
	
	
	\subsection*{[实验目的]}
    \par 1.了解迈克尔孙干涉仪的结构原理并掌握调节方法。
    \par 2.观察等厚干涉、等倾干涉以及白光干涉。
	\par 3.测量钠双线的波长差。
	\subsection*{[实验器材]}
	\par 迈克尔逊干涉仪,He-Ne多光束光纤激光器。
\subsection*{[实验原理]}
\subsubsection*{迈克耳孙干涉仪的结构}
迈克尔逊干涉仪是分振幅法的双光束干涉仪,由反射镜\(M_1\)、\(M_2\),分束镜\(P_1\)和补偿板\(P_2\)组成。\(M_1\)固定,\(M_2\)可沿光轴移动,二者置于相互垂直臂中;分束镜和补偿板与反射镜均成\(45^{\circ}\)且相互平行,分束镜\(P_1\)一面镀半透半反膜,能将入射光等强度分为两束,补偿板与分束镜厚度和折射率相同。

\begin{figure}[!htbp]\centering\includegraphics[width=0.5\textwidth]{figures/图片1.png}\caption{}\end{figure}

多光束激光器光纤输出端经短焦距凸透镜会聚的激光束可看作点光源发出的球面光波,\(S_1'\)为\(S\)经\(M_1\)及\(G_1\)反射后所成的像,\(S_2'\)为\(S\)经\(G_1\)及\(M_2\)反射后所成的像,\(S_1'\)和\(S_2'\)为相干光源,发出的球面波在相遇空间处处相干,为非定域干涉。

\begin{figure}
[H]\centering\includegraphics[width=0.3\textwidth]{figures/图片2.png}\caption{}\end{figure}


空间任一点\(P\)的干涉明暗由\(S_2'\)和\(S_1'\)到该点的光程差\(\Delta = r_2 - r_1\)决定,\(P\)点光强分布的极大和极小条件为:
\[
\begin{align*}
\Delta &= k\lambda, \text{ 亮条纹}\\
\Delta &= (2k + 1)\lambda, \text{ 暗条纹}
\end{align*}
\]
\begin{figure}
[H]\centering\includegraphics[width=0.2\textwidth]{figures/图片3.png}\caption{}\end{figure}
当\(M_1'\)与\(M_2\)平行时,观察屏垂直\(S_2'S_1'\)连线放置,可看到同心干涉圆条纹。设\(M_1'\)与\(M_2\)间距离为\(d\),则\(S_2'\)和\(S_1'\)距离为\(2d\),\(S_2'\)和\(S_1'\)在屏上任一点的光程差为\(\Delta = 2d\cos\varphi\),\(\varphi\)为\(S_2'\)射到\(P\)点的光线与\(M_2\)法线的夹角。

\subsubsection*{条纹数量
N和d改变量的关系}

当\(d\)增加\(\frac{\lambda}{2}\),光程差增加\(\lambda\),中心条纹干涉级次由\(k\)变为\(k + 1\),会“冒出”一个条纹;当\(d\)减少\(\frac{\lambda}{2}\),光程差减少\(\lambda\),中心条纹干涉级次由\(k\)变为\(k - 1\),会“缩进”一个条纹。

根据“冒出”或“缩进”条纹的个数\(N\)可确定\(d\)的改变量\(\delta d=\frac{N\lambda}{2}\),可用于长度测量,精度达波长量级。 

\subsection*{[实验内容]}
\subsubsection*{调节干涉仪,观察非定域干涉}
\par \textbf{一、水平调节}\quad 调节干涉仪底脚螺丝,使仪器导轨平面水平,然后用锁紧圈锁住。
\par \textbf{二、等臂调节}\quad 调节粗调手轮移动$M_2$镜,让$M_1,M_2$镜与分光板$G_1$,大致等距离。
\par \textbf{三、最亮点重合}\quad 打开激光开关,检查激光输出嘴的位置和方向,让光束垂直射向$M_1$的中心部位。将观察屏转向一侧并周定,戴上墨镜,直接观察$M_2$镜,视野 中呈现两排分别由$M_1$、$M_2$反射回来的亮点,找准每排亮点中最亮的那个点,分别调 节$M_1$和$M_2$两个反射镜背后的调节螺华(先调$M_1$再调$M_2$),使两排亮点中最亮的 光点严格重合,此时说明$M_1$已垂直$M_2$。注意调节时调节螺丝的松紧要均衡,防止损坏调节螺丝。
\par \textbf{四、条纹移到屏中央}\quad 将观察屏转回原位置,若上一步中的最亮点已经严格重合,则观察屏上可以观察到圆形干涉条纹,若没有条纹,可能是亮点没严格重合,或者条纹在屏幕边缘。调节粗调手轮使条纹大小、粗细适中,再轻微调节$M_1$,镜上的水平或竖直拉簧螺丝,使圆形条纹的中心位于屏中央。
\par \textbf{五、观察非定域干涉}\quad 前后左右移动屏的位置和角度,发现干涉条纹的大小或形状发生变化,证明非定义域干涉是空间处处相干的。
\par \textbf{六、条纹特征与$d$的关系} \quad 调节粗调手轮前后移动M2,观察条纹的“冒出”或“缩进”现象,判断M;与M之间的距离d是变大还是变小,并观察条纹的粗细、疏密和d之间的关系。

\subsubsection*{测量激光波长}
\par \textbf{一、仪器调零} \quad 因为旋转微调手轮时,粗调手轮随之变化,而旋转粗调手轮时
微调手轮并不随之变化,所以测量前必须调零。方法如下:沿某方向(例如顺时针)
将微调手轮调到零并记住旋转方向(为避免空程差,后面的测量都要沿此方向),沿
同一方向旋转粗调手轮使之对准某一刻度,注意此后粗调手轮不要再动。测量过
程中若需要反方向旋转微调手轮,则一定要重新调零。
\par \textbf{二、测量并计算波长} \quad 沿刚才的方向旋转微调手轮,条纹每冒出或缩进50个记录相应的M2的位置,连续记录6次以上,数据记录在表4-5-1中,用最小二乘法计算激光的波长。
\clearpage
\subsection*{[数据处理]}


\begin{table}[h]
\centering
\begin{tabular}{|c|c|c|c|c|c|c|}
\hline
条纹移动数N1 & 0 & 50 & 100 & 150 & 200 & 250 \\
\hline
可移动镜位置$d1/mm$ & 65.10120 & 65.11700 & 65.13290 & 65.14803 & 65.16432 & 65.18044 \\
\hline
$\Delta d/mm$ &  & 0.01580 &  & 0.01513 &  & 0.01612 \\
\hline
\end{tabular}
\caption{实验数据}
\end{table}

\begin{figure}
[H]\centering\includegraphics[width=0.7\textwidth]{figures/Figure_1.png}\caption{关系图}\end{figure}

\subsubsection*{使用最小二乘法计算波长}
我们可以使用最小二乘法计算斜率 \(k\)。最小二乘法中对于线性关系 \(y = kx + b\),斜率 \(k\) 的计算公式为:

\[k=\frac{n\sum_{i = 1}^{n}x_iy_i-\sum_{i = 1}^{n}x_i\sum_{i = 1}^{n}y_i}{n\sum_{i = 1}^{n}x_i^2-(\sum_{i = 1}^{n}x_i)^2}\]

已知 \(x\) 为条纹移动数 \(N_1\),\(y\) 为可移动镜位置 \(d_1\),\(n = 6\)。

\(N_1=\{0, 50, 100, 150, 200, 250\}\),\(d_1=\{65.10120, 65.11700, 65.13290, 65.14803, 65.16432, 65.18044\}\)

首先计算相关求和项:
\par 1. \(\sum_{i = 1}^{n}x_i=0 + 50+100 + 150+200+250 = 750\)
\par 2. \(\sum_{i = 1}^{n}y_i=65.10120 + 65.11700+65.13290+65.14803+65.16432+65.18044 = 390.84389\)
\par 3. \(\sum_{i = 1}^{n}x_i^2=0^2 + 50^2+100^2 + 150^2+200^2+250^2=0 + 2500+10000+22500+40000+62500 = 137500\)
\par 4. \(\sum_{i = 1}^{n}x_iy_i=0\times65.10120+50\times65.11700 + 100\times65.13290+150\times65.14803+200\times65.16432+250\times65.18044\)
   \(=0 + 3255.85+6513.29+9772.2045+13032.864+16295.11\)
   \(=48869.3185\)

将上述值代入斜率公式可得:

\[
\begin{align*}
k&=\frac{n\sum_{i = 1}^{n}x_iy_i-\sum_{i = 1}^{n}x_i\sum_{i = 1}^{n}y_i}{n\sum_{i = 1}^{n}x_i^2-(\sum_{i = 1}^{n}x_i)^2}\\
&=\frac{6\times48869.3185 - 750\times390.84389}{6\times137500-750^2}\\
&=\frac{293215.911 - 293132.9175}{825000 - 562500}\\
&=\frac{82.9935}{262500}\\
&\approx0.00031617
\end{align*}
\]

而 \[\lambda = 2 \frac{\Delta d}{\Delta N} = 2 k = 632.3 nm\]

相对误差:
\[
\eta = \frac{|\lambda - \lambda_{real}|}{\lambda_{real}} \times 100\% =  0.07\%
\]

\subsection*{[思考题]}

\par 1、补偿板 G2 的作用是什么?
\par 答:在迈克尔逊干涉仪中,补偿板 G2 与分束镜的厚度和折射率完全相同。其主要作用是补偿光路。从光源发出的光经分束镜分为两束,一束经反射镜 M1 反射,另一束经反射镜 M2 反射。由于经分束镜分光后,两束光所经过的玻璃介质厚度不同(其中一束光会多次经过分束镜) ,会产生额外的光程差。补偿板 G2 使两束光经过相同厚度的玻璃介质,保证两束光之间的光程差只由反射镜 M1、M2 位置等因素决定,消除玻璃介质带来的额外光程差影响,确保干涉实验的准确性。 
\par 2、为什么改变Δd 只能朝一个方向? 
\par 答:主要是为了避免空程误差。在实验中,仪器的传动部件(如丝杆等)存在间隙。如果改变Δd 时来回转动,在改变方向时,丝杆要先消除间隙才能真正带动反射镜移动,这就会导致读数与实际移动距离不对应,产生较大误差。始终朝一个方向改变Δd ,可以保证读数与实际移动距离一致,提高实验测量的精度 。
\par 3、如何确定反射镜 M1 与 M2 是互相垂直的?
\par 答:两排亮点严格重合。
    在迈克尔逊干涉仪实验中,当 M1 与 M2 互相垂直时,会观察到同心圆形的等倾干涉条纹。具体操作是,先调节粗调手轮和微调手轮,观察干涉条纹的变化。当缓慢调节反射镜 M2 的位置和角度时,若出现一组清晰、稳定的同心圆形干涉条纹,且圆心位于视场中心附近,就可以认为反射镜 M1 与 M2 互相垂直 。

\subsection*{[实验总结]}

\par 本次迈克尔逊干涉仪实验,通过操作仪器观察干涉条纹,利用原理测量相关量,加深了对干涉现象及仪器应用的理解。 在粗调时应该格外注意,干涉要达到“稳定”。


\end{document}
