\documentclass[12pt,a4paper,UTF8]{ctexart}




%设置页边距
\usepackage{geometry}
\geometry{left=2.5cm,right=2.5cm,top=2.5cm,bottom=2.5cm}
\usepackage{wrapfig}



%需要用到的扩展包
\usepackage{xeCJK,amsmath,paralist,enumerate,booktabs,multirow,graphicx,float,subfig,setspace,listings,lastpage,hyperref}
\usepackage{fancyhdr}




%设置页眉页脚以及页码
\pagestyle{fancy}
\rhead{衍射光栅}
\lhead{大学基础物理实验报告}
\rfoot{2025 年 4 月 29 日 }




%报告中用到的图片存放在这个tex文件所在目录中的figures子目录中
\graphicspath{{figures/}}









%报告开始
\begin{document}
	
	
	
	
	%设置课程标题
	\begin{center}
		\heiti\LARGE{《大学基础物理实验》课程实验报告}
	\end{center}
	
	
	
	
	%设置实验人信息以及实验时间表格
	
	
		\begin{center}
			\begin{tabular}{lcr}
				
				{\songti 姓名:柯云超  \quad 学号: 2413575}\\
				{\songti  学院:计算机学院 \quad 时间:2025年4月29日 \quad 组别: L组11号}\\
				
				
			\end{tabular}
		\end{center}
	\vspace{-0.2cm}
	{\noindent}	 \rule[-10pt]{16cm}{0.05em}\\

	\vspace{-0.4cm}
	
	
	
	
	
	
	%实验题目
	\begin{center}
		\LARGE\textbf{衍射光栅}
	\end{center}
	
	
	\subsection*{[实验目的]}
    \par 1.了解光栅的分光特性。
    \par 2.测量光栅常量。
	\subsection*{[实验器材]}
	

\par 分光仪,平面投射光栅,平面反射镜,低压汞灯


\subsection*{[实验原理]}
二元光栅是平行等宽、等间距的多狭缝,它的分光原理如图所示。狭缝S处于透镜\(L_1\)的焦平面上,并认为它是无限细的;G是衍射光栅,它有N个宽度为a的狭缝,相邻狭缝间不透明部分的宽度为b。如果自透镜\(L_1\)出射的平行光垂直照射在光栅上,透镜\(L_2\)将与光栅法线成\(\theta\)角的光会聚在焦平面上的P点。光栅在\(\theta\)方向上有主干涉极大的条件为
\begin{figure}[!htbp]
	\centering
	\includegraphics[width=0.9\textwidth]{figures/光栅原理.png}
	\caption{光栅原理}
\end{figure}

\[
(a + b)\sin\theta=k\lambda
\]
这就是垂直入射条件下的光栅方程,式中,\(k\)为光谱的级次、\(\lambda\)是波长、\(\theta\)是衍射角、\((a + b)\)是光栅常量。光栅常量通常用\(d\)表示,\(d = a + b\)。

当入射光不是垂直照射在光栅上,而是与光栅的法线成\(\varphi\)角时,光栅方程变为
\[
d(\sin\varphi\pm\sin\theta)=k\lambda
\]
式中“+”代表入射光和衍射光在法线同侧,“ - ”代表在法线两侧。光栅的衍射角\(\theta\)仍定义为与光栅表面法线的夹角。

在复色光以相同的入射角照射到光栅,不同波长的光对应有不同的\(\theta\)角,也就是说在经过光栅后,不同波长的光在空间角方向上被分开了,并按一定的顺序排列。这就是光栅的分光原理。 




\subsection*{[实验内容]}
\subsubsection*{调节分光仪}
\par 按上次实验的方法调节分光仪到可以使用的状态。
\subsubsection*{调节光栅}
\par 实验中的光栅必须调节到以下状态。
\par (1)平行光垂直照射在光栅表面。
\par (2)光栅的刻痕垂直于刻度盘表面,即与仪器转轴平行。
\par (3)狭缝与光栅刻痕平行。
\begin{figure}[!htbp]
	\centering
	\includegraphics[width=0.3\textwidth]{figures/光栅位置.png}
	\caption{光栅位置}
\end{figure}
\clearpage
将光栅按图1所示的方式放置在载物台上。光栅平面与$V_1,V_3$的连线垂直。用汞灯照亮狭缝,使望远镜的叉丝对准狭缝像。这样望远镜的光轴与平行光管的光轴共线。将游标盘与载物台锁定在一起,转动载物台,找到平面光栅反射回来的叉丝像,调节$V_1,V_3$使叉丝像与叉丝重合,随即锁住游标盘,并保持$V_1,V_3$不动。这时就达到了光栅与入射的平行光垂直的要求。此时转动望远镜观察位于零级谱两侧的一级或二级谱线,调节$V_2$和稍微旋转狭缝,使两侧谱线均与叉丝的中心横线垂直,并且上下对称。这时光栅就已经调节好了。
\subsubsection*{误差来源及解决办法}
\par 实验所用的透射光栅是做在一个全息干板上,全息干版的两个面不可能完全平行,因此无论怎样都不可能让入射光线完全垂直与光栅表面。在斜入射的情况下,光栅法线两侧同一级光谱的衍射角分别为
$$
\left.\begin{array}{l}
	\sin \varphi-\sin \theta_{-}=-\dfrac{k \lambda}{d} \\
	\sin \varphi+\sin \theta_{+}=\dfrac{k \lambda}{d}
\end{array}\right\}
$$
\par 两式相减,并考虑$|\theta_{+}-\theta_{-}|=\varphi$
\[\sin{\frac{\theta_{+}-\theta_{-}}{2}}\cos{\frac{\varphi}{2}}=\frac{k \lambda}{d}\]
\par 当$\varphi$很小的时,$\sin{\dfrac{\theta_{+}-\theta_{-}}{2}}=\dfrac{k \lambda}{d}$
\subsubsection*{测量数据}
\par 利用汞光谱线中绿线$\lambda = 546.1\quad nm$的$\pm 1,\pm 2$级光谱之间的夹角$2\theta_1,2\theta_2$,分别求出两个光栅常量。


\subsection*{[数据处理]}
% Please add the following required packages to your document preamble:
% \usepackage{multirow}
% Please add the following required packages to your document preamble:
% \usepackage{multirow}
\par 根据公式$\sin{\dfrac{\theta_{+}-\theta_{-}}{2}}=\dfrac{k \lambda}{d}$计算得:
\begin{table}[!htbp]
	\centering
	\begin{tabular}{c|c|ccc|c|c|c}
		\hline
		\multirow{2}{*}{波长}    & \multirow{2}{*}{级数} & \multicolumn{3}{c|}{衍射角位置}                                   & \multirow{2}{*}{角度} & \multirow{2}{*}{无偏心角度}  & \multirow{2}{*}{光栅常量} \\ \cline{3-5}
		&                     & \multicolumn{1}{c|}{游标号} & \multicolumn{1}{c|}{+k级}    & -k级     &                     &                         &                       \\ \hline
		\multirow{4}{*}{546.1 nm} & \multirow{2}{*}{1}  & \multicolumn{1}{c|}{1}   & \multicolumn{1}{c|}{$9^{\circ}24^{\prime}$} & $9^{\circ}28^{\prime}$   & $18^{\circ}54^{\prime}$               & \multirow{2}{*}{$18^{\circ}54^{\prime}30^{\prime\prime}$} & \multirow{2}{*}{3324.64nm}     \\ \cline{3-6}
		&                     & \multicolumn{1}{c|}{2}   & \multicolumn{1}{c|}{$9^{\circ}28^{\prime}$} & $9^{\circ}27^{\prime}$  & $18^{\circ}55^{\prime}$                &                         &                       \\ \cline{2-8} 
		& \multirow{2}{*}{2}  & \multicolumn{1}{c|}{1}   & \multicolumn{1}{c|}{$19^{\circ}7^{\prime}$} & $19^{\circ}7^{\prime}$ & $38^{\circ}14^{\prime}$               & \multirow{2}{*}{$38^{\circ}28^{\prime}$} & \multirow{2}{*}{3315.56nm}     \\ \cline{3-6}
		&                     & \multicolumn{1}{c|}{2}   & \multicolumn{1}{c|}{$19^{\circ}5^{\prime}$} & $19^{\circ}37^{\prime}$    & $38^{\circ}42^{\prime}$                &                         &                       \\ \hline
	\end{tabular}	
\end{table}
\par 所以$\overline{d}=3320.1nm$,光栅刻痕密度 301 条/mm。
\clearpage
% Please add the following required packages to your document preamble:
% \usepackage{multirow}
\begin{table}[!htbp]
	\centering
	\begin{tabular}{c|c|ccc|c|c|c}
		\hline
		\multirow{2}{*}{汞黄线} & \multirow{2}{*}{级数} & \multicolumn{3}{c|}{衍射角位置}                                                                        & \multirow{2}{*}{角度}     & \multirow{2}{*}{无偏心角度}                   & \multirow{2}{*}{波长} \\ 
		\cline{3-5}
		&                     & \multicolumn{1}{c|}{游标号} & \multicolumn{1}{c|}{k}                       & k                       &                         &                                          &                       \\ 
		\hline
		\multirow{2}{*}{黄1}  & \multirow{2}{*}{2}  & \multicolumn{1}{c|}{1}   & \multicolumn{1}{c|}{$20^{\circ}08^{\prime}$} & $20^{\circ}50^{\prime}$ & $40^{\circ}58^{\prime}$ & \multirow{2}{*}{$40^{\circ}48^{\prime}$} & \multirow{2}{*}{570.5nm}     \\ 
		\cline{3-6}
		&                     & \multicolumn{1}{c|}{2}   & \multicolumn{1}{c|}{$19^{\circ}52^{\prime}$} & $20^{\circ}40^{\prime}$ & $40^{\circ}38^{\prime}$ &                                          &                       \\ 
		\hline
		\multirow{2}{*}{黄2}  & \multirow{2}{*}{2}  & \multicolumn{1}{c|}{1}   & \multicolumn{1}{c|}{$20^{\circ}15^{\prime}$} & $20^{\circ}40^{\prime}$ & $40^{\circ}55^{\prime}$ & \multirow{2}{*}{$40^{\circ}58^{\prime}$} & \multirow{2}{*}{572.7nm}     \\ 
		\cline{3-6}
		&                     & \multicolumn{1}{c|}{2}   & \multicolumn{1}{c|}{$20^{\circ}11^{\prime}$} & $20^{\circ}50^{\prime}$ & $41^{\circ}01^{\prime}$ &                                          &                       \\ 
		\hline
	\end{tabular}
\end{table}
\par 跟标准值$\lambda_1 = 577.0nm,\lambda_2 = 579.1nm$计算得到误差为:
\[ \Delta \lambda_1 = \frac{|577.0-570.5|}{577.0}=0.011  \]
\[\Delta \lambda_1 = \frac{|579.1-572.7|}{579.1}=0.011\]
\par 而角色散
\[D = \frac{\Delta \varphi}{2.1nm} = \frac{10^{\prime}}{2.1nm} = \frac{\frac{10^{\prime}}{180^{\circ}}\times \pi}{2.1nm} = 1.358\times10^{-3}\times rad\times nm^{-1}\]


\subsection*{[思考题]}
1. 对测量衍射角的影响及解决方法
 
影响:仪器转轴未通过光栅平面时,左右两侧衍射光线不对称,导致测量的衍射角偏差,结果不准确。
解决方法:用自准直法调整,通过调节载物台螺丝,使光栅平面反射的十字像与望远镜上十字叉丝重合,确保光栅平面与转轴平行且过轴。

2.根据光栅方程
\begin{equation}
d\sin\theta = k\lambda
\end{equation}
对于钠黄光的两个波长 \(\lambda_1 = 589.0\ nm\) 和 \(\lambda_2 = 589.6\ nm\),在同一级次 \(k\) 下,衍射角分别为 \(\theta_1\) 和 \(\theta_2\)。
由于在成像物镜焦平面上,衍射角较小时,\(\tan\theta\approx\sin\theta\),设成像物镜焦距为 \(f\),两个波长在焦平面上分开的距离 \(\Delta x\) 满足:
\begin{equation}
\Delta x = f(\sin\theta_2 - \sin\theta_1)
\end{equation}
由光栅方程可得 \(\sin\theta_1=\frac{k\lambda_1}{d}\),\(\sin\theta_2=\frac{k\lambda_2}{d}\),则
\begin{equation}
\Delta x = \frac{kf(\lambda_2 - \lambda_1)}{d}
\end{equation}

已知 \(\Delta x = 1\ mm = 1\times10^{-3}\ m\),\(\lambda_2 - \lambda_1=(589.6 - 589.0)\times10^{-9}\ m = 0.6\times10^{-9}\ m\)。
假设选择 \(k = 1\)(一级衍射),若选取常见的成像物镜焦距 \(f = 1\ m\),由 \(\Delta x = \frac{kf(\lambda_2 - \lambda_1)}{d}\) 可得:
\begin{align*}
d&=\frac{kf(\lambda_2 - \lambda_1)}{\Delta x}\\
&=\frac{1\times1\times0.6\times10^{-9}}{1\times10^{-3}} \\
&= 6\times10^{-7}\ m = 600\ nm
\end{align*}

\par 钠光灯:提供钠黄光(含 \(589.0\ nm\) 和 \(589.6\ nm\) 两个波长)。
\par 狭缝:限制光束宽度,形成线光源。
\par 准直透镜:将从狭缝出射的光变为平行光,假设焦距为 \(f_1\)。 
\par 光栅:光栅常数 \(d = 600\ nm\),对钠黄光进行衍射分光。
\par 成像物镜:焦距 \(f = 1\ m\),将衍射光会聚到焦平面。
\par 观察屏:放置在成像物镜焦平面处,用于观察两个波长分开的像。
\begin{figure}[!htbp]
	\centering
	\includegraphics[width=0.9\textwidth]{figures/光路图.png}
	\caption{光路图}
\end{figure}


\end{document}
