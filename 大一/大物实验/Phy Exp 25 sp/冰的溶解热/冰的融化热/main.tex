\documentclass[12pt,a4paper,UTF8]{ctexart}
%设置页边距
\usepackage{geometry}
\geometry{left=2.5cm,right=2.5cm,top=2.5cm,bottom=2.5cm}
\usepackage{wrapfig}
%需要用到的扩展包
\usepackage{xeCJK,amsmath,paralist,enumerate,booktabs,multirow,graphicx,float,subfig,setspace,listings,lastpage,hyperref}
\usepackage{fancyhdr}
\usepackage{siunitx}

\pagestyle{fancy}
\rhead{用混合量热法测定冰的溶化热}
\lhead{大学基础物理实验报告}
\rfoot{2025年6月3日}







%报告开始
\begin{document}
	%设置课程标题
	\begin{center}
		\heiti\LARGE{《大学基础物理实验》课程实验报告}
	\end{center}
	
	
	
	
	%设置实验人信息以及实验时间表格
	
	
		\begin{center}
			\begin{tabular}{lcr}
				
				{\songti 姓名:柯云超  \quad 学号: 2413575}\\
				{\songti  学院:计算机学院 \quad 时间:2025年6月3日 \quad 组别: L组11号}\\
				
				
			\end{tabular}
		\end{center}
	\vspace{-0.2cm}
	{\noindent}	 \rule[-10pt]{16cm}{0.05em}\\

	\vspace{-0.4cm}


	%实验题目
	\begin{center}
		\LARGE\textbf{用混合量热法测定冰的溶化热}
	\end{center}
    
	
\subsection*{[实验目的]}
1.正确使用量热器,熟练使用温度计

2.用混合量热法测定冰的熔化热

3.进行实验安排和参量选取

4.学会一种粗略修正散热的方法------抵偿法



\subsection*{[实验器材]}
\par 量热器,温度计,电子天平,秒表,玻璃皿,干布,保温桶,冰以及热水等。

\subsection*{[实验原理]}

设法制造一个与外界无热量交换的孤立系统C(只包含冰块A与已知热容的系统B),满足热平衡方程如下:

\[Q_\text{吸}=Q_\text{放}\]
对于B,则有
\[Q=C_s\mathrm{\Delta\theta}\]

质量$m_i$,温度$\theta_0^\prime$(本次实验为0度) 的冰块与质量m,温度$\theta_1$的水相结合,冰全部熔化为水,测得平衡温度为$\theta_2$。假定量热容器内筒与搅拌器的质量分别为$m_1,m_2$,其比热容分别为$c_1,c_2$,数字式温度计的测温传感器(铂电阻测温探头)比热容很小,可以忽略不计;水和冰的比热容分别为c和$c_i$(其中水的比热容为$c=4.1868kJ\cdot\ kg^{-1}\cdot\ K^{-1} ,c_i=1.8kJ\cdot\ kg^{-1}\cdot\ K^{-1})$;记冰的熔点为$\theta_0$,本次实验$\theta_0$为0摄氏度.

可以得到:

\begin{equation}
   L=\frac{1}{m_i}\left(cm+c_1m_1+c_2m_2\right)\left(\theta_1-\theta_2\right)-c\theta_2 
\end{equation}

为减少与外界热交换带来的实验误差,我们以抵偿法作为的散热修正方法。原理如下

设环境温度为$\theta_e$,当系统的温度高于环境温度时,它就要散失热量。由当温差较小(一般不超过15 K)时,(非自然对流)系统的散热制冷速率与温差成正比,此时牛顿冷却定律表示为:

\begin{equation}
    \frac{dq}{dx}=-k\left(\theta-\theta_e\right)
\end{equation}

整个过程的温度变化曲线如下:

\begin{figure}[!htbp]
	\centering
	\includegraphics*[width=0.6\textwidth]{img1.jpg}
\end{figure}

如果使\[S_A=S_B\]则前后热交换相抵消,则在这个过程中可以近似视为孤立系统

总而言之,我们可知通过适当地选择参数,使曲线与环境温度$\theta=\theta_e$,直线围成的两块面积近似相等,就可以使系统很好地近似为一个孤立系统。

\subsection*{[实验步骤]}


\begin{enumerate}
    \item 使用温度计与电子天平测量环境温度\(\theta_{e}\) ,以及内桶质量\(m_1\)和搅拌器质量\(m_2\) 。
    \item 向内筒中倒入适量的温水(水温比环境温度高约\(10\)到\(15\)摄氏度最好),测量\(m + m_1+m_2\) 。 
    \item 将量热桶放入装置,盖好盖子 ,持续监测温度\(4\)分钟,每分钟记录一次温度数值。 
    \item 准备冰块,用干布擦去冰表面的水分,在第\(5\)分钟时将冰块投入内筒,迅速盖上盖子并搅拌,每\(10\)秒钟测量一次温度,直至系统温度开始上升。测量搅拌器、量热器和冰与水的总质量\(m + m_1+m_2 + m_i\)  。 
    \item 绘制\(\theta - t\)曲线,据此求出冰的溶解热,重复上述步骤三次
\end{enumerate}

\subsection*{[数据处理]}

记录得到如下数据,其中0 - 240s是单独测水温的变化,300s是投入冰块的时刻。

\renewcommand{\arraystretch}{1.2}

\begin{table}[htbp]
\centering
\begin{tabular}{|c|c|c|c|c|c|c|c|}
\hline
时间 & 温度 & 时间 & 温度 & 时间 & 温度 & 时间 & 温度 \\
\hline
0s & 36.7℃ & 340s & 31.4℃ & 430s & 25.5℃ & 520s & 24.4℃ \\
\hline
60s & 36.4℃ & 350s & 30.2℃ & 440s & 25.4℃ & 530s & 24.3℃ \\
\hline
120s & 36.3℃ & 360s & 29.1℃ & 450s & 25.4℃ & 540s & 24.3℃ \\
\hline
180s & 36.2℃ & 370s & 28.2℃ & 460s & 25.2℃ & 550s & 24.3℃ \\
\hline
240s & 36.0℃ & 380s & 27.5℃ & 470s & 25.0℃ & 560s & 24.3℃ \\
\hline
300s & 35.9℃ & 390s & 26.9℃ & 480s & 24.9℃ & 570s & 24.3℃ \\
\hline
310s & 34.7℃ & 400s & 26.5℃ &490s & 24.7℃ & 580s & 24.3℃ \\
\hline
320s & 33.6℃ & 410s & 26.0℃ & 500s & 24.6℃ & 590s & 24.3℃ \\
\hline
330s & 32.6℃ & 420s & 25.6℃ & 510s & 24.5℃ & 600s & 24.3℃ \\
\hline

\hline
\end{tabular}
\label{tab:measurement}
\end{table}

使用python对于上述数据进行绘制得到Figure1.如下

\begin{figure}[!htbp]
	\centering
	\includegraphics*[width=0.95\textwidth]{Figure_1.png}
\end{figure}

其中灰色点代表原始数据,曲线代表拟合曲线。紫色虚线表示室温,紫色与绿色表示曲线与室温直线所围城的面积

通过观察,取紫色部分与绿色部分,二者面积相近。可以将系统近似视为孤立系统。可以得到
$\theta_2=24.3$摄氏度.

而其他测量数据如下:

\begin{align*}
        \begin{aligned}
        m_2=8.64g\\
        m_1+m_2=116.30g\\
        m_1+m_2+m=344.57g\\
        m_1+m_2+m+m_i=361.77g \\
        \end{aligned}
\end{align*}

再其他数据带入公式公式(1)得到\[L_\text{测}\approx3.167\times10^5Jkg^{-1}\]

而\[L_\text{0}=3.248\times10^5Jkg^{-1}\]

计算得到的误差为

\[\eta=\left|\frac{L_{测}-L_0}{L_0}\right|\times100%\approx2.5\%\]

\subsection*{[误差分析]}

经过测量发现与真实值存在一定的误差,分析原因可能有以下几个:

1.放入冰块时,由于操作不熟练,导致放入冰块所花时间较长,可能有一部分热量散失了。但是L偏小

2.图像中S1与S2的面积可能不是严格相等的,而是S1略大。即有一部分热量散失了。导致L偏小。

3.冰块未完全处于熔点(0℃),融化时需先吸收热量升温,若实验模型假设冰块初始温度为 0℃,则会低估实际所需热量 Q,导致 L 偏小。

\subsection*{[思考题]}
\begin{enumerate}
    % 题目1
    \item 假如冰内有①气泡、②小水泡、③杂质,它们分别对实验结果有无影响?为什么?
    \begin{itemize}
        \item 气泡:有影响。冰熔化后气泡逸出,测量的冰质量(含气泡体积对应的“虚假”质量 )大于实际参与熔化的冰质量。根据 \( Q = mL \)(\( Q \) 为熔化吸热,\( m \) 为冰质量,\( L \) 为熔化热 ),\( m \) 偏大,计算出的 \( L \) 偏小 。
        \item 小水泡:有影响。小水泡内水是“提前存在的液态水”,使冰熔化需吸收的热量测量值偏小(部分热量并非用于冰熔化为水 )。由 \( L = Q/m \),\( Q \) 偏小、\( m \) 含小水泡对应冰的“虚假”质量,会导致计算的 \( L \) 偏小 。 
        \item 杂质:若杂质不参与热交换,测量的冰质量(含杂质 )大于实际冰质量,根据 \( L = Q/m \),\( m \) 偏大,\( L \) 偏小;若杂质参与热交换(如熔化吸热 ),会使 \( Q \) 测量偏大,\( L \) 可能偏大,一般实验未考虑杂质热性质时,多使 \( L \) 测量不准确,常因冰质量测量易偏大导致 \( L \) 偏小 。 
    \end{itemize}

    % 题目2
    \item 如果冰中含水量为 \( x\% \),试求由此引起冰熔化热 \( L \) 的相对误差。

    
         设冰质量为 \( m \),纯冰熔化热 \( L_0 \),含 \( x\% \) 水的冰可视为 \( m(1 - x\%) \) 冰与 \( mx\% \) 水。实际冰熔化吸热 \( Q = m(1 - x\%)L_0 \),测量时按总质量 \( m \) 算,得 \( L = \frac{Q}{m} = (1 - x\%)L_0 \) 。
        
         相对误差 \( \eta = \frac{|L - L_0|}{L_0} = \frac{|(1 - x\%)L_0 - L_0|}{L_0} = x\% \)(理想情况,假设水不影响吸热测量 )。 


    % 题目3
    \item 若给定 \( L_0 = 3.341×10^5 \, \mathrm{J·kg^{-1}} \),试求 \( L \) 的定值误差。

    
\[
|\Delta L| = |L_{\text{测}} - L_0|
\]
代入数据:
\[
|\Delta L| = |3.167 \times 10^5 - 3.248 \times 10^5| = 8.1 \times 10^3 \, \text{J/kg}
\]

\[
\delta = \left| \frac{L_{\text{测}} - L_0}{L_0} \right| \times 100\%
\]
代入数据:
\[
\delta = \left| \frac{3.167 \times 10^5 - 3.248 \times 10^5}{3.248 \times 10^5} \right| \times 100\% = \frac{8.1 \times 10^3}{3.248 \times 10^5} \times 100\% \approx 2.49\%
\]
\end{enumerate}
\section*{附录}

原始实验数据:

\begin{figure}[!htbp]
	\centering
	\includegraphics*[width=0.95\textwidth]{data.jpg}
\end{figure}


\end{document}